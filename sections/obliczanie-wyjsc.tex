Nieodzownym krokiem treningu sieci neuronowej jest obliczenie aktualnych wyników i porównanie ich z oczekiwanymi.
Przedstawmy to na przykładzie wejścia (0,1), którego oczekiwanym wyjściem jest 0.
Informację o wejściu podajemy neuronom wejściowym, tak więc: \(i1=0\), \(i2=1\).

Aby obliczyć wartości kolejnych neuronów musimy zsumować wartości połączeń do tego neuronu, a następnie poddając wynik funkcji aktywacji.
Przez wartość połączenia będziemy rozumieli iloczyn wagi i neuronu wejściowego dla połączenia.
Wagi połączeń zostały przedstawione na rysunku \ref{fig:separability}.

Przykładowo dla neuronu h1 wartość przed użyciem funkcji aktywacji będzie równa:
\[
  net_{h1}  = i1*w11+i2*w21
\]
Co po podstawieniu wartości da:
\[
  net_{h1} = 0*0.5+1*0.1=0.1
\]
Taka wartość jest następnie poddawana działaniu funkcji aktywacji, 
co umożliwia normalizowanie wartości neuronów do oczekiwanych przedziałów wartości.
W tym przypadku skorzystamy z funkcji sigmoidalnej, której zakres wartości mieści się w przedziale [0,1]:
\[
    f(x) =  \frac{1}{1 + e^{-x} } 
\]
Po podstawieniu otrzymanej wcześniej wartości do funkcji dostaniemy wartość neuronu h1:
\[
    out_{h1} 
    = f(net_{h1}) 
    =  \frac{1 }{1 + e^{-net_{h1}} } 
    =  \frac{1 }{1 + e^{-0.1} } \approx 0.525
\]
W ten sposób będziemy liczyć wartość kolejnych neuronów:
\[
    out_{h2}=f(i1*w12+i2*w22)=f(0*0.3+1*0.8)=f(0.8) \approx 0.690
\]
\[
    out_{o}=f(h1*w1o+h2*w2o)=f(0.525*0.4+0.690*0.9)=f(0.831) \approx 0.697
\]
